\documentclass[10pt,a4paper]{article}
\usepackage[utf8]{inputenc}
\usepackage{amsmath}
\usepackage{amsfonts}
\usepackage{amssymb}
\usepackage[T1]{fontenc}
\begin{document}
\begin{LARGE}
\begin{center}\textbf{Projekcik}\\ \textbf{Krzysztof Ksi\k{a}\.zek}\end{center}
\end{LARGE}Utworzyc plik \textit{pliktex02.tex} . Napisac funkcj\k{e}, ktora jako argument przyjmie wektor (ci\k{a}g) macierzy i zapisze w pliku \textit{pliktex02.tex} macierze maj\k{a}ce rzeczywiste wartosci w\l{}asne oraz wektory w\l{}asne tych macierzy. Wspo\l{}rz\k{e}dne wektorow w\l{}asnych zaokr\k{a}glic do jednego miejsca po przecinku w gore. Sprawdzic mo\.zliwosc kompilacji pliku \textit{pliktex02.tex}.\\ \\
\noindent\rule[0.5cm]{\textwidth}{1pt}
\textbf{Macierz  1 :}\\
$$\begin{pmatrix} 1 & 2 & 0 \\ 0 & 2 & 0 \\ -2 & -2 & -1 \\ \end{pmatrix}$$
Wartosci w\l{}asne: 2 -1 1 \\Dla wartosci w\l{}asnej  $\lambda =  2 $ \\
Przyporz\k{a}dkowywujemy nast\k{e}pujce wektory w\l{}asne: \
$$ \begin{pmatrix}  0.7 & 0.4 & -0.6 \\ \end{pmatrix}$$
Dla wartosci w\l{}asnej  $\lambda =  -1 $ \\
Przyporz\k{a}dkowywujemy nast\k{e}pujce wektory w\l{}asne: \
$$ \begin{pmatrix}  0 & 0 & 1 \\ \end{pmatrix}$$
Dla wartosci w\l{}asnej  $\lambda =  1 $ \\
Przyporz\k{a}dkowywujemy nast\k{e}pujce wektory w\l{}asne: \
$$ \begin{pmatrix}  0.8 & 0 & -0.7 \\ \end{pmatrix}$$
\noindent\rule[0.5cm]{\textwidth}{1pt}
\textbf{Macierz  2 :}\\
$$\begin{pmatrix} 3 & 2 \\ 4 & 1 \\ \end{pmatrix}$$
Wartosci w\l{}asne: 5 -1 \\Dla wartosci w\l{}asnej  $\lambda =  5 $ \\
Przyporz\k{a}dkowywujemy nast\k{e}pujce wektory w\l{}asne: \
$$ \begin{pmatrix}  0.8 & 0.8 \\ \end{pmatrix}$$
Dla wartosci w\l{}asnej  $\lambda =  -1 $ \\
Przyporz\k{a}dkowywujemy nast\k{e}pujce wektory w\l{}asne: \
$$ \begin{pmatrix}  -0.4 & 0.9 \\ \end{pmatrix}$$
\noindent\rule[0.5cm]{\textwidth}{1pt}
\textbf{Macierz  3 :}\\
$$\begin{pmatrix} 0 & 1 \\ -1 & 0 \\ \end{pmatrix}$$
Wartosci wlasne oraz wektory wlasne nie sa liczbami rzeczywistymi\end{document}